% !TeX root = RJwrapper.tex
\title{Rain in Australia. Classification Prediction Model}
\author{by Sumaira Afzal, Viraja Ketkar, Murlidhar Loka, Vadim Spirkov}

\maketitle

\abstract{%
An abstract of less than 150 words.
}

% Any extra LaTeX you need in the preamble

\hypertarget{introduction}{%
\subsection{Introduction}\label{introduction}}

\hypertarget{background}{%
\subsection{Background}\label{background}}

\hypertarget{objective}{%
\subsection{Objective}\label{objective}}

\hypertarget{data-analisys}{%
\section{Data Analisys}\label{data-analisys}}

The data set we are going to use for our research contains daily weather
observations from numerous Australian weather stations from 2007 till
2017. There are over 142000 records. It has been sourced from
\href{https://www.kaggle.com/jsphyg/weather-dataset-rattle-package}{Kagle}

\hypertarget{data-dictionary}{%
\subsection{Data Dictionary}\label{data-dictionary}}

We exclude the variable Risk-MM when training your binary classification
model. If we don't exclude it, you will leak the answers to our model
and reduce its predictability

\begin{longtable}[]{@{}ll@{}}
\toprule
\begin{minipage}[b]{0.52\columnwidth}\raggedright
Column Name\strut
\end{minipage} & \begin{minipage}[b]{0.43\columnwidth}\raggedright
Column Description\strut
\end{minipage}\tabularnewline
\midrule
\endhead
\begin{minipage}[t]{0.52\columnwidth}\raggedright
Date\strut
\end{minipage} & \begin{minipage}[t]{0.43\columnwidth}\raggedright
Date of observation\strut
\end{minipage}\tabularnewline
\begin{minipage}[t]{0.52\columnwidth}\raggedright
Location\strut
\end{minipage} & \begin{minipage}[t]{0.43\columnwidth}\raggedright
Common name of the location of the weather station\strut
\end{minipage}\tabularnewline
\begin{minipage}[t]{0.52\columnwidth}\raggedright
MinTemp\strut
\end{minipage} & \begin{minipage}[t]{0.43\columnwidth}\raggedright
Minimum temperature in degrees celsius\strut
\end{minipage}\tabularnewline
\begin{minipage}[t]{0.52\columnwidth}\raggedright
MaxTemp\strut
\end{minipage} & \begin{minipage}[t]{0.43\columnwidth}\raggedright
Maximum temperature in degrees celsius\strut
\end{minipage}\tabularnewline
\begin{minipage}[t]{0.52\columnwidth}\raggedright
Rainfall\strut
\end{minipage} & \begin{minipage}[t]{0.43\columnwidth}\raggedright
Amount of rainfall recorded for the day in mm\strut
\end{minipage}\tabularnewline
\begin{minipage}[t]{0.52\columnwidth}\raggedright
Evaporation\strut
\end{minipage} & \begin{minipage}[t]{0.43\columnwidth}\raggedright
So-called Class A pan evaporation (mm) in the 24 hours to 9am\strut
\end{minipage}\tabularnewline
\begin{minipage}[t]{0.52\columnwidth}\raggedright
Sunshine\strut
\end{minipage} & \begin{minipage}[t]{0.43\columnwidth}\raggedright
Number of hours of bright sunshine in the day\strut
\end{minipage}\tabularnewline
\begin{minipage}[t]{0.52\columnwidth}\raggedright
WindGustDir\strut
\end{minipage} & \begin{minipage}[t]{0.43\columnwidth}\raggedright
Direction of the strongest wind gust in the 24 hours to midnight\strut
\end{minipage}\tabularnewline
\begin{minipage}[t]{0.52\columnwidth}\raggedright
WindGustSpeed\strut
\end{minipage} & \begin{minipage}[t]{0.43\columnwidth}\raggedright
Speed (km/h) of the strongest wind gust in the 24 hours to
midnight\strut
\end{minipage}\tabularnewline
\begin{minipage}[t]{0.52\columnwidth}\raggedright
WindDir9amDirection\strut
\end{minipage} & \begin{minipage}[t]{0.43\columnwidth}\raggedright
Of the wind at 9am\strut
\end{minipage}\tabularnewline
\begin{minipage}[t]{0.52\columnwidth}\raggedright
WindDir3pmDirection\strut
\end{minipage} & \begin{minipage}[t]{0.43\columnwidth}\raggedright
Of the wind at 3pm\strut
\end{minipage}\tabularnewline
\begin{minipage}[t]{0.52\columnwidth}\raggedright
WindSpeed9amWind\strut
\end{minipage} & \begin{minipage}[t]{0.43\columnwidth}\raggedright
Wind speed (km/hr) averaged over 10 minutes prior to 9am\strut
\end{minipage}\tabularnewline
\begin{minipage}[t]{0.52\columnwidth}\raggedright
WindSpeed3pmWind\strut
\end{minipage} & \begin{minipage}[t]{0.43\columnwidth}\raggedright
Wind Speed (km/hr) averaged over 10 minutes prior to 3pm\strut
\end{minipage}\tabularnewline
\begin{minipage}[t]{0.52\columnwidth}\raggedright
Humidity9amHumidity\strut
\end{minipage} & \begin{minipage}[t]{0.43\columnwidth}\raggedright
Humidity (percent) at 9am\strut
\end{minipage}\tabularnewline
\begin{minipage}[t]{0.52\columnwidth}\raggedright
Humidity3pmHumidity\strut
\end{minipage} & \begin{minipage}[t]{0.43\columnwidth}\raggedright
Humidity (percent) at 3pm\strut
\end{minipage}\tabularnewline
\begin{minipage}[t]{0.52\columnwidth}\raggedright
Pressure9amAtmospheric\strut
\end{minipage} & \begin{minipage}[t]{0.43\columnwidth}\raggedright
Pressure (hpa) reduced to mean sea level at 9am\strut
\end{minipage}\tabularnewline
\begin{minipage}[t]{0.52\columnwidth}\raggedright
Pressure3pmAtmospheric\strut
\end{minipage} & \begin{minipage}[t]{0.43\columnwidth}\raggedright
Pressure (hpa) reduced to mean sea level at 3pm\strut
\end{minipage}\tabularnewline
\begin{minipage}[t]{0.52\columnwidth}\raggedright
Cloud9amFraction\strut
\end{minipage} & \begin{minipage}[t]{0.43\columnwidth}\raggedright
Area of sky obscured by cloud at 9am. This is measured in ``oktas'',
which are a unit of eigths. It records how many eigths of the sky are
obscured by cloud. A 0 measure indicates completely clear sky whilst an
8 indicates that it is completely overcast\strut
\end{minipage}\tabularnewline
\begin{minipage}[t]{0.52\columnwidth}\raggedright
Cloud3pmFraction\strut
\end{minipage} & \begin{minipage}[t]{0.43\columnwidth}\raggedright
Area of sky obscured by cloud (in ``oktas'': eighths) at 3pm. See
Cload9am for a description of the values\strut
\end{minipage}\tabularnewline
\begin{minipage}[t]{0.52\columnwidth}\raggedright
Temp9amTemperature\strut
\end{minipage} & \begin{minipage}[t]{0.43\columnwidth}\raggedright
Temperature (degrees C) at 9am\strut
\end{minipage}\tabularnewline
\begin{minipage}[t]{0.52\columnwidth}\raggedright
Temp3pmTemperature\strut
\end{minipage} & \begin{minipage}[t]{0.43\columnwidth}\raggedright
Temperature (degrees C) at 3pm\strut
\end{minipage}\tabularnewline
\begin{minipage}[t]{0.52\columnwidth}\raggedright
RainTodayBoolean\strut
\end{minipage} & \begin{minipage}[t]{0.43\columnwidth}\raggedright
Rainy today. 1 if precipitation (mm) in the 24 hours to 9am exceeds 1mm,
otherwise 0\strut
\end{minipage}\tabularnewline
\begin{minipage}[t]{0.52\columnwidth}\raggedright
RISK\_MM\strut
\end{minipage} & \begin{minipage}[t]{0.43\columnwidth}\raggedright
Amount of rain. A kind of measure of the ``risk''. This column is
redundant and will be dropped\strut
\end{minipage}\tabularnewline
\begin{minipage}[t]{0.52\columnwidth}\raggedright
\textbf{RainTomorrowThe}\strut
\end{minipage} & \begin{minipage}[t]{0.43\columnwidth}\raggedright
\textbf{Target variable. Will it rain tomorrow?}\strut
\end{minipage}\tabularnewline
\bottomrule
\end{longtable}

\hypertarget{data-exploration}{%
\subsection{Data Exploration}\label{data-exploration}}

Let's take a close look at the data set. We start with loading weather
observations from the file into a data frame. We remove RISK\_MM as
explained and convert Date column to \emph{date} format

\begin{Schunk}
\begin{Sinput}
weatherData = read.csv("../data/weatherAUS.csv", header = TRUE, na.strings = c("NA","","#NA"),sep=",")
weatherData = subset(weatherData, select = -RISK_MM)
weatherData$Date = as.Date(as.character(weatherData$Date),"%Y-%m-%d")
\end{Sinput}
\end{Schunk}

Now let's load coordinates of the weather stations and have a bird-eye
view of the weather station locations

\begin{Schunk}
\begin{figure}

{\centering \includegraphics[width=1.1\linewidth]{images/weatherStations} 

}

\caption[Australian Weather Stations]{Australian Weather Stations}\label{fig:map}
\end{figure}
\end{Schunk}

\newpage

Let's review data summary

\begin{Schunk}
\begin{Sinput}
summary(weatherData)
\end{Sinput}
\begin{Soutput}
#>       Date                Location         MinTemp         MaxTemp     
#>  Min.   :2007-11-01   Canberra:  3418   Min.   :-8.50   Min.   :-4.80  
#>  1st Qu.:2011-01-06   Sydney  :  3337   1st Qu.: 7.60   1st Qu.:17.90  
#>  Median :2013-05-27   Perth   :  3193   Median :12.00   Median :22.60  
#>  Mean   :2013-04-01   Darwin  :  3192   Mean   :12.19   Mean   :23.23  
#>  3rd Qu.:2015-06-12   Hobart  :  3188   3rd Qu.:16.80   3rd Qu.:28.20  
#>  Max.   :2017-06-25   Brisbane:  3161   Max.   :33.90   Max.   :48.10  
#>                       (Other) :122704   NA's   :637     NA's   :322    
#>     Rainfall       Evaporation        Sunshine      WindGustDir   
#>  Min.   :  0.00   Min.   :  0.00   Min.   : 0.00   W      : 9780  
#>  1st Qu.:  0.00   1st Qu.:  2.60   1st Qu.: 4.90   SE     : 9309  
#>  Median :  0.00   Median :  4.80   Median : 8.50   E      : 9071  
#>  Mean   :  2.35   Mean   :  5.47   Mean   : 7.62   N      : 9033  
#>  3rd Qu.:  0.80   3rd Qu.:  7.40   3rd Qu.:10.60   SSE    : 8993  
#>  Max.   :371.00   Max.   :145.00   Max.   :14.50   (Other):86677  
#>  NA's   :1406     NA's   :60843    NA's   :67816   NA's   : 9330  
#>  WindGustSpeed      WindDir9am      WindDir3pm     WindSpeed9am 
#>  Min.   :  6.00   N      :11393   SE     :10663   Min.   :  0   
#>  1st Qu.: 31.00   SE     : 9162   W      : 9911   1st Qu.:  7   
#>  Median : 39.00   E      : 9024   S      : 9598   Median : 13   
#>  Mean   : 39.98   SSE    : 8966   WSW    : 9329   Mean   : 14   
#>  3rd Qu.: 48.00   NW     : 8552   SW     : 9182   3rd Qu.: 19   
#>  Max.   :135.00   (Other):85083   (Other):89732   Max.   :130   
#>  NA's   :9270     NA's   :10013   NA's   : 3778   NA's   :1348  
#>   WindSpeed3pm    Humidity9am      Humidity3pm      Pressure9am    
#>  Min.   : 0.00   Min.   :  0.00   Min.   :  0.00   Min.   : 980.5  
#>  1st Qu.:13.00   1st Qu.: 57.00   1st Qu.: 37.00   1st Qu.:1012.9  
#>  Median :19.00   Median : 70.00   Median : 52.00   Median :1017.6  
#>  Mean   :18.64   Mean   : 68.84   Mean   : 51.48   Mean   :1017.7  
#>  3rd Qu.:24.00   3rd Qu.: 83.00   3rd Qu.: 66.00   3rd Qu.:1022.4  
#>  Max.   :87.00   Max.   :100.00   Max.   :100.00   Max.   :1041.0  
#>  NA's   :2630    NA's   :1774     NA's   :3610     NA's   :14014   
#>   Pressure3pm        Cloud9am        Cloud3pm        Temp9am     
#>  Min.   : 977.1   Min.   :0.00    Min.   :0.0     Min.   :-7.20  
#>  1st Qu.:1010.4   1st Qu.:1.00    1st Qu.:2.0     1st Qu.:12.30  
#>  Median :1015.2   Median :5.00    Median :5.0     Median :16.70  
#>  Mean   :1015.3   Mean   :4.44    Mean   :4.5     Mean   :16.99  
#>  3rd Qu.:1020.0   3rd Qu.:7.00    3rd Qu.:7.0     3rd Qu.:21.60  
#>  Max.   :1039.6   Max.   :9.00    Max.   :9.0     Max.   :40.20  
#>  NA's   :13981    NA's   :53657   NA's   :57094   NA's   :904    
#>     Temp3pm      RainToday     RainTomorrow
#>  Min.   :-5.40   No  :109332   No :110316  
#>  1st Qu.:16.60   Yes : 31455   Yes: 31877  
#>  Median :21.10   NA's:  1406               
#>  Mean   :21.69                             
#>  3rd Qu.:26.40                             
#>  Max.   :46.70                             
#>  NA's   :2726
\end{Soutput}
\end{Schunk}

\hypertarget{missing-data}{%
\subsubsection{Missing Data}\label{missing-data}}

Further analisys of data shows that many features are missing. Some data
losses are very significant. We are going to identify what data is
missing and if it is feasible to recover the data.

\begin{Schunk}
\begin{Sinput}
print(sort(colSums(is.na(weatherData)), decreasing = T))
\end{Sinput}
\begin{Soutput}
#>      Sunshine   Evaporation      Cloud3pm      Cloud9am   Pressure9am 
#>         67816         60843         57094         53657         14014 
#>   Pressure3pm    WindDir9am   WindGustDir WindGustSpeed    WindDir3pm 
#>         13981         10013          9330          9270          3778 
#>   Humidity3pm       Temp3pm  WindSpeed3pm   Humidity9am      Rainfall 
#>          3610          2726          2630          1774          1406 
#>     RainToday  WindSpeed9am       Temp9am       MinTemp       MaxTemp 
#>          1406          1348           904           637           322 
#>          Date      Location  RainTomorrow 
#>             0             0             0
\end{Soutput}
\end{Schunk}

To speed up data processing and plot rendering we are going to use a
data sample. For population of 142K obesrvations, 20K sample size would
be sufficient for 99\% confidence level with the condience interval 1

\begin{Schunk}
\begin{Sinput}
weatherSample = sample_n(weatherData, 20000)
aggr(weatherSample, numbers = F, prop = T, col = mainPalette, sortVars = T, bars = F, varheight = T)
\end{Sinput}
\begin{figure}

{\centering \includegraphics{source_files/figure-latex/plot_aggr_missing-1} 

}

\caption[Missing Data Summary]{Missing Data Summary}\label{fig:plot_aggr_missing}
\end{figure}
\begin{Soutput}
#> 
#>  Variables sorted by number of missings: 
#>       Variable   Count
#>       Sunshine 0.47255
#>    Evaporation 0.42295
#>       Cloud3pm 0.39845
#>       Cloud9am 0.37580
#>    Pressure9am 0.09835
#>    Pressure3pm 0.09760
#>     WindDir9am 0.07000
#>    WindGustDir 0.06365
#>  WindGustSpeed 0.06325
#>     WindDir3pm 0.02765
#>    Humidity3pm 0.02475
#>   WindSpeed3pm 0.01880
#>        Temp3pm 0.01870
#>    Humidity9am 0.01315
#>   WindSpeed9am 0.00995
#>       Rainfall 0.00990
#>      RainToday 0.00990
#>        Temp9am 0.00705
#>        MinTemp 0.00455
#>        MaxTemp 0.00195
#>           Date 0.00000
#>       Location 0.00000
#>   RainTomorrow 0.00000
\end{Soutput}
\end{Schunk}

As demonstrated in Figure \ref{fig:plot_aggr_missing} \emph{Sunshine},
\emph{Evaporation} and \emph{Clouds} columns saffer the loss of data
between \textbf{48\%} and \textbf{38\%}. This is significant! Sinse we
are dealing with the weather patterns we should be observing cyclical
data pattenrs. Let's review data stribution of features that damaged the
most.

\begin{Schunk}
\begin{figure}

{\centering \includegraphics{source_files/figure-latex/plot_margin1-1} 

}

\caption[Date/Evaporation Margin Plot]{Date/Evaporation Margin Plot}\label{fig:plot_margin1}
\end{figure}
\end{Schunk}

\begin{Schunk}
\begin{figure}

{\centering \includegraphics{source_files/figure-latex/plot_margin2-1} 

}

\caption[Date/ Sunshine Margin Plot]{Date/ Sunshine Margin Plot}\label{fig:plot_margin2}
\end{figure}
\end{Schunk}

\begin{Schunk}
\begin{figure}

{\centering \includegraphics{source_files/figure-latex/plot_margin3-1} 

}

\caption[Date/ Pressure3pm Margin Plot]{Date/ Pressure3pm Margin Plot}\label{fig:plot_margin3}
\end{figure}
\end{Schunk}

\newpage

So what the margin plot tell us? First of all let's take a look at
\emph{Date} axis. The \emph{Date} has been converted to number to ensure
continuous flow of the data . All features we picked exhibit cyclical
pattern as expected. Along the vertical axis we obesrve the box plot of
the respective feature. \emph{Evaporaton} data is quite remarkable; it
has very narrow distribution and a lot outliers.

\hypertarget{data-preparation}{%
\subsection{Data Preparation}\label{data-preparation}}

\hypertarget{modeling-and-evalutation}{%
\section{Modeling and Evalutation}\label{modeling-and-evalutation}}

\hypertarget{decision-tree-model}{%
\subsection{Decision Tree Model}\label{decision-tree-model}}

\hypertarget{naive-bayes-model}{%
\subsection{Naive Bayes Model}\label{naive-bayes-model}}

\hypertarget{random-forest-model}{%
\subsection{Random Forest Model}\label{random-forest-model}}

\hypertarget{logistic-regression-model}{%
\subsection{Logistic Regression Model}\label{logistic-regression-model}}

\hypertarget{model-comparison}{%
\subsection{Model Comparison}\label{model-comparison}}

\hypertarget{model-deployment}{%
\section{Model Deployment}\label{model-deployment}}

\hypertarget{conclusion}{%
\section{Conclusion}\label{conclusion}}

\hypertarget{bibliography}{%
\section{Bibliography}\label{bibliography}}


\address{%
Sumaira Afzal\\
York University School of Continuing Studies\\
\\
}


\address{%
Viraja Ketkar\\
York University School of Continuing Studies\\
\\
}


\address{%
Murlidhar Loka\\
York University School of Continuing Studies\\
\\
}


\address{%
Vadim Spirkov\\
York University School of Continuing Studies\\
\\
}


